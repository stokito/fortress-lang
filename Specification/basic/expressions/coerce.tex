%%%%%%%%%%%%%%%%%%%%%%%%%%%%%%%%%%%%%%%%%%%%%%%%%%%%%%%%%%%%%%%%%%%%%%%%%%%%%%%%
%   Copyright 2009 Sun Microsystems, Inc.,
%   4150 Network Circle, Santa Clara, California 95054, U.S.A.
%   All rights reserved.
%
%   U.S. Government Rights - Commercial software.
%   Government users are subject to the Sun Microsystems, Inc. standard
%   license agreement and applicable provisions of the FAR and its supplements.
%
%   Use is subject to license terms.
%
%   This distribution may include materials developed by third parties.
%
%   Sun, Sun Microsystems, the Sun logo and Java are trademarks or registered
%   trademarks of Sun Microsystems, Inc. in the U.S. and other countries.
%%%%%%%%%%%%%%%%%%%%%%%%%%%%%%%%%%%%%%%%%%%%%%%%%%%%%%%%%%%%%%%%%%%%%%%%%%%%%%%%

\subsection{Coercion}
\seclabel{coercion-expr}

An identity function \EXP{\mathrm{coerce}} is defined in \library\
to convert the type of its argument to its type argument:
%% coerce_[\T\](x: T) = x
\begin{Fortress}
\(\mathrm{coerce}\llbracket{}T\rrbracket(x\COLON T) = x\)
\end{Fortress}
The function returns its argument as the given type.
Unlike coercions described in \chapref{conversions-coercions},
the \EXP{\mathrm{coerce}} function can apply to an argument whose type is
a subtype of the type being coerced to.
